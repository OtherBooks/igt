
\section{Two-Player Non-Zero-Sum Games}\label{pdandchicken}

Before getting any further into non-zero-sum games, let's recall some key ideas about zero-sum games. 
\begin{itemize}
\item If a zero-sum game has an equilibrium point, then repeating the game does not affect how the players will play. 
\item If a zero-sum game has more that one equilibrium point then the values of the equilibrium points are the same.
\item In a zero-sum game, we can find mixed strategy equilibrium points using the graphical method or the expected value method.
\item In a zero-sum game, a player never benefits from communicating her strategy to her opponent.
\end{itemize}

Hopefully, in the last section you saw that non-zero-sum games can differ on all of the above!

\begin{example}\label{E:simplenonzero}
Let's  consider the game given by Table \ref{T:simplenonzero}.
\begin{table}[h]
\centering
\begin{tabular}{rcc}
&\textbf{C}&\textbf{D}\\ 
\textbf{A} &(0, 0)&(10, 5) \\ 
\textbf{B}&(5, 10)&(0, 0) \\ 
\end{tabular}
\caption{A non-zero sum example}
\label{T:simplenonzero}
\end{table}
\end{example}

\begin{xca}\label{E:simplenzero}
Check that this is NOT a zero-sum game. 
\end{xca}

\begin{xca}\label{E:simplefindequil}
Using the ``guess and check'' method for finding equilibria, you should be able to determine that there are two equilibrium points. What are they? 
\end{xca}

\begin{xca}\label{E:simpleprefer}
As we saw in Section \ref{Intrononzero}, non-zero-sum games equilibrium points need not have the same values. Does Player 1 prefer one of these equilibria over the other?
\end{xca}

Since it is now possible for BOTH players to benefit at the same time, it might be a good idea for players to communicate with each other. For example, if Player 1 says that she will choose A no matter what, then it is in Player 2's best interest to choose D. If communication is allowed in the game, then we say the non-zero-sum game is \emph{cooperative}\index{cooperative game}. If no communication is allowed, we say it is \emph{non-cooperative}\index{non-cooperative game}. 

We saw in Section \ref{Intrononzero}, that our methods for analyzing zero-sum games do not work very well on non-zero-sum games. Let's look a little closer at this. 

If we apply the graphical method for Player 1 to the above game, we get that Player 1 should play a (1/3, 2/3) mixed strategy for an expected payoff of 10/3. Similarly we can determine that Player 2 should play a (2/3, 1/3) mixed strategy for an expected payoff of 10/3. Recall we developed this strategy as a ``super defensive'' strategy. But are our players motivated to play as defensively in a non-zero-sum game? Not necessarily! It is no longer true that Player 2 needs to keep Player 1 from gaining! 

Now suppose, Player 1 plays the (1/3, 2/3) strategy. Then the expected payoff to Player 2 for playing pure strategy C, $E_2(C)$, is 20/3; and the expected payoff to Player 2 for playing pure strategy D, $E_2(D)$, is 5/3. Thus Player 2 prefers C over D. But if Player 2 plays only C, then Player 1 should abandon her (1/3, 2/3) strategy and just play B! This results in the payoff vector (5, 10). Notice, that now the expected value for Player 1 is 5, which is better than 10/3! Again, since Player 2 is not trying to keep Player 1 from gaining, there is no reason to apply the maximin strategy to non-zero-sum games. Similarly, we don't want to apply the expected value solution since Player 1 does not care if Player 2's expected values are equal. Each player only cares about his or her own payoff, not the payoff of the other player.

OK, so now, how do we analyze these games? 

\begin{xca}\label{E:conjgeneralstrat}
What are some possible strategies for each player? Might some strategies depend on what a player knows about her opponent?
\end{xca}

Can you see that some of the analysis might be better understood with psychology than with mathematics? 



